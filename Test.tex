\documentclass{article}
\usepackage[utf8]{inputenc}
\usepackage[T1]{fontenc}
\usepackage[utf8]{inputenc}
\usepackage{lmodern}
\usepackage[a4paper, margin=1in]{geometry}
\usepackage{graphicx}
\graphicspath{ {./image/} }
\usepackage{minted}

% For footer and header -
\usepackage{fancyhdr}
\pagestyle{fancy}
\fancyhf{}
\fancyfoot[L]{\textsc{ARYAN PARATAKKE}}
\fancyfoot[R]{\textsc{22070521070}}
\fancyhead[R]{\thepage} % line for page number in header
\renewcommand{\footrulewidth}{0.4pt} % Add a horizontal line above the footer

% Begin document
\begin{document}
    \begin{titlepage}
        \begin{center}
            \huge{\bfseries \textbf{SYMBIOSIS INSTITUTE OF TECHNOLOGY,  NAGPUR.}}\\ 
        [1.5cm]
        % For image
            \begin{center}
                \includegraphics[width=6cm]{image.png}\\
        [1.5cm] 
                \textsc{\huge{Computer Science and Engineering\\Batch 2022-26}}\\  
        [0.5cm]
        % For line
                \line(1,0){350}\\
        [0.65cm]
            \end{center}   
        \huge{\bfseries Fundamentals of Data Structure}\\
        \line(1,0){350}\\
        [0.5cm]
            \textsc{\Large CA-3: Case Study}\\
            \textsc{\LARGE {October, 2023}}\\  
        [1.5cm]
            \textsc{\LARGE{Aryan Paratakke}}\\
            \textsc{\Large prn: 22070521070}\\
        \end{center}		
    \end{titlepage}

\tableofcontents % For contents page
\pagebreak

% Start with sections
\section{Case Study: Implementing a Stack Using Queues}
    \subsection{Objective}
        The goal of this assignment is to test your understanding of data structures, specifically stacks and queues, by challenging you to implement a stack using queues.
        
    \subsection{Background}\
        You have learned about stacks and queues, two fundamental data structures with different principles: Last-In-First-Out (LIFO) for stacks and First-In-First-Out (FIFO) for queues. This assignment aims to reinforce your knowledge and problem-solving skills by implementing a stack using queues.
    
    \subsection{Thought Process}

        
    \subsection{Program}

        \begin{minted}[frame=lines, linenos, fontsize=\large]
        {c}
        #include <stdio.h>
        int main()
        {
            int A1[] = {1, 3, 5};
            int A2[] = {2, 4, 6, 8, 11, 13};
            int C[20];
            int i = 0, j = 0, k = 0;
            int n1 = sizeof(A1) / sizeof(A1[0]), n2 = sizeof(A2) / sizeof(A2[0]);
            while (i < n1 && j < n2)
            {
                if (A1[i] <= A2[j])
                    C[k++] = A1[i++];
                else
                    C[k++] = A2[j++];
            }
            while (i < n1)
                C[k++] = A1[i++];
            while (j < n2)
                C[k++] = A2[j++];
            printf("Array 1 : ");
        
            for (int m = 0; m < n1; m++)
                printf("%d ", A1[m]);
            printf("\nArray 2 : ");
        
            for (int m = 0; m < n2; m++)
                printf("%d ", A2[m]);
            printf("\nMerged array : ");
        
            int n3 = k;
            for (int m = 0; m < n3; m++)
                printf("%d ", C[m]);
            printf("\nTotal number of elements in merged array = %d", n3);
            return 0;
        }
        \end{minted}
        
    \subsection{Output}

\pagebreak

\section{Case Study: Implementing a Stack Using Queues}
    \subsection{Objective}
        The goal of this assignment is to test your understanding of data structures, specifically stacks and queues, by challenging you to implement a stack using queues.
        
    \subsection{Background}\
        You have learned about stacks and queues, two fundamental data structures with different principles: Last-In-First-Out (LIFO) for stacks and First-In-First-Out (FIFO) for queues. This assignment aims to reinforce your knowledge and problem-solving skills by implementing a stack using queues.
    
    \subsection{Thought Process}

        
    \subsection{Program}

        \begin{minted}[frame=lines, linenos, fontsize=\large]
        {c}
        #include <stdio.h>
        int main()
        {
            int A1[] = {1, 3, 5};
            int A2[] = {2, 4, 6, 8, 11, 13};
            int C[20];
            int i = 0, j = 0, k = 0;
            int n1 = sizeof(A1) / sizeof(A1[0]), n2 = sizeof(A2) / sizeof(A2[0]);
            while (i < n1 && j < n2)
            {
                if (A1[i] <= A2[j])
                    C[k++] = A1[i++];
                else
                    C[k++] = A2[j++];
            }
            while (i < n1)
                C[k++] = A1[i++];
            while (j < n2)
                C[k++] = A2[j++];
            printf("Array 1 : ");
        
            for (int m = 0; m < n1; m++)
                printf("%d ", A1[m]);
            printf("\nArray 2 : ");
        
            for (int m = 0; m < n2; m++)
                printf("%d ", A2[m]);
            printf("\nMerged array : ");
        
            int n3 = k;
            for (int m = 0; m < n3; m++)
                printf("%d ", C[m]);
            printf("\nTotal number of elements in merged array = %d", n3);
            return 0;
        }
        \end{minted}
        
    \subsection{Output}


\pagebreak

\section{Case Study: Implementing a Stack Using Queues}
    \subsection{Objective}
        The goal of this assignment is to test your understanding of data structures, specifically stacks and queues, by challenging you to implement a stack using queues.
        
    \subsection{Background}\
        You have learned about stacks and queues, two fundamental data structures with different principles: Last-In-First-Out (LIFO) for stacks and First-In-First-Out (FIFO) for queues. This assignment aims to reinforce your knowledge and problem-solving skills by implementing a stack using queues.
    
    \subsection{Thought Process}

        
    \subsection{Program}

        \begin{minted}[frame=lines, linenos, fontsize=\large]
        {c}
        #include <stdio.h>
        int main()
        {
            int A1[] = {1, 3, 5};
            int A2[] = {2, 4, 6, 8, 11, 13};
            int C[20];
            int i = 0, j = 0, k = 0;
            int n1 = sizeof(A1) / sizeof(A1[0]), n2 = sizeof(A2) / sizeof(A2[0]);
            while (i < n1 && j < n2)
            {
                if (A1[i] <= A2[j])
                    C[k++] = A1[i++];
                else
                    C[k++] = A2[j++];
            }
            while (i < n1)
                C[k++] = A1[i++];
            while (j < n2)
                C[k++] = A2[j++];
            printf("Array 1 : ");
        
            for (int m = 0; m < n1; m++)
                printf("%d ", A1[m]);
            printf("\nArray 2 : ");
        
            for (int m = 0; m < n2; m++)
                printf("%d ", A2[m]);
            printf("\nMerged array : ");
        
            int n3 = k;
            for (int m = 0; m < n3; m++)
                printf("%d ", C[m]);
            printf("\nTotal number of elements in merged array = %d", n3);
            return 0;
        }
        \end{minted}
        
    \subsection{Output}


\pagebreak

\end{document}
